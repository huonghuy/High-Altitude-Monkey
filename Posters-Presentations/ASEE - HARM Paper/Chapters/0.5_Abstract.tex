\textbf{Abstract} - This Work In Progress paper is about a high altitude ballooning payload involving long distance robotics communications. In 2018, as part of a NASA-funded student project, undergraduate students developed a robotic monkey which successfully operated on a 2000-gram high-altitude balloon (HAB) and flew to an altitude of 4,000 feet and a range of 12 miles. The robot, named HAM (for “High Altitude Monkey”), moves, speaks, and broadcasts live video while in flight. Its purpose was to interact with K-12 students on the ground as an educational tool to teach “Near Space” science. That robot is now part of the university’s mobile laboratory (STEM bus), which travels to local area schools and community events around the state, promoting near space research and STEM education.

In the academic year 2024-2025, a new group of interdisciplinary engineering and computer science students will build an enhanced robot and resume its high-altitude ballooning flights. A major enhancement to the robot’s flight control payload is the addition of a controllable and autonomous vent. The vent is attached to the high-altitude balloon itself and can be opened or closed to release helium, allowing the balloon to remain at a specified altitude and enabling altitude-controlled termination of the flight. Additionally, the vent system will enable the payload to stay near the ground station site, supporting better live stream transmission and reception.

The target for this project is to launch HAM to 15,000 feet and travel approximately 20 miles while maintaining a direct live feed for the entire flight. A team of undergraduate and graduate students will have the opportunity to perform fundamental research in developing this payload. The successful completion of HAM will support the use of robotics in near-space applications as well as research on long-distance communications and transmissions. The outcomes from this project will be disseminated through demonstrations on the university’s mobile laboratory for K-12 students and through peer-reviewed publications.
    
\textbf{Keywords}

    high-altitude, balllooning, robotics

