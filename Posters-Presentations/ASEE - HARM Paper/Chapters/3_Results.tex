\vspace{-2em}
\section{Result \& Discussion}
\vspace{-2em}

The preliminary findings from this research reflect not only strong technical progress toward a successful flight in June 2025, but also highlight the educational impact of interdisciplinary skill-building and the broader outreach potential of HAM as an interactive STEM platform.

To date, students have successfully established APRS communication with the payload at distances of 20 and 200 feet, verifying baseline functionality between the visual payload, telemetry, and APRS. The system continues to perform reliably and remains on track for full integration by the end of May 2025. Additional validation through a preliminary tethered launch is planned to assess data transmission capabilities under simulated flight conditions.

Notably, neither of HAM’s lead students comes from a traditional background in mechanical or electrical engineering—disciplines upon which this renovation heavily relies. Despite this, both Huong and Gudi, having previously participated in high-altitude ballooning missions, earned their degrees in Computer Science and took the initiative to independently acquire the interdisciplinary skills required for this project. These skills span soldering, 3D modeling, schematic design, PCB layout, and project management. Their technical contributions have been instrumental to the project’s success and have significantly enriched their academic growth and professional readiness as they prepare to enter industry roles.

Unlike many ballooning payloads that operate passively with minimal engagement, HAM introduces a unique interactive component. By responding to real-time commands during flight, HAM creates opportunities for live engagement with students and observers on the ground. This interactivity enables us to teach foundational STEM principles—such as robotics, telemetry, and embedded systems—to learners of all ages via our STEM on Wheels program, the University's mobile STEM laboratory. Just as the original HAM the Chimpanzee inspired a generation in 1961 \cite{Burgess2014}, our goal is for this modern-day HAM to spark curiosity and inspire the next generation of scientists and engineers through meaningful, hands-on experiences.