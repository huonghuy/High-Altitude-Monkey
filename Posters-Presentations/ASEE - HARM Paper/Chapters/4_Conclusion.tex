\vspace{-2em}
\section{Future Outlooks and Considerations}
\vspace{-2em}

High-altitude ballooning has enabled researchers to conduct iterative experiments with rapid turnaround, offering a flexible and cost-effective alternative to traditional astronomical research. HAM was originally designed to engage K–12 students through interactive, hands-on science experiences, and our team is committed to continuing this mission. Through the University of Bridgeport's \textit{STEM on Wheels} program—a mobile laboratory that brings STEM activities directly to schools across Connecticut—we aim to inspire students and educators by demonstrating real-world applications of science and engineering.

To further encourage adoption and innovation, all design files and documentation for HAM will be made publicly available via GitHub. This will allow other institutions to replicate our work, build their own high-altitude flight companions, and contribute to a growing community focused on improving communication systems in unmanned stratospheric flights.

The success of the HAM project is the result of contributions from students, faculty, and alumni spanning multiple generations and disciplines. Looking ahead, the university plans to establish HAM as an annual project embedded within intercollegiate clubs—providing interdisciplinary teams of undergraduate and graduate students with the opportunity to apply their skills and propose system-level improvements. Areas currently under review for future revisions include:

\vspace{-2em}
\begin{itemize}
    \item Integration of DC-DC power conversion units and reconfiguration of power distribution systems
    \item Use of the ATmega2560 microcontroller with a direct programming and communication interface
    \item Replacement of proprietary communication modules with onboard Bluetooth
    \item Substitution of the discontinued EMIC 2 Text-to-Speech module with a modern alternative
    \item Enhancement of servo motor performance through the integration of Bottango software for unique "Monkey-like" motions \cite{Bottango}
\end{itemize}
\vspace{-2em}

This High Altitude Robotic Monkey exemplifies the power of interdisciplinary learning, hands-on engineering, and public outreach through high-altitude ballooning. From its roots as a student-driven initiative to its evolution into an interactive STEM platform, HAM has provided valuable educational experiences, technical challenges, and outreach opportunities. As the project concludes its 2025 iteration, it lays the foundation for continued development, regular deployment, and broader adoption by academic institutions seeking to engage students in space-related research. With open-source documentation, long-term university support, and integration into community outreach, HAM is positioned to serve as both a model for experiential learning and a catalyst for future innovation in stratospheric flight.